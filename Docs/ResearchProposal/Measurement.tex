

\subsection{Sample Preparation}
\subsubsection*{Timeline}
\begin{itemize}
 	\item Completed Tasks
	\begin{itemize}
		\item Mount and measure test femur
	\end{itemize}
	\item Current Tasks (Expected to be operational by October Beamtime)
	\begin{itemize}
		\item Mount and measure the extreme cases to ensure measurement process works
		\item Determine maximum error in mounting where alignment algorithms and measurement still functions
		\item Glue and Mount 200 samples
	\end{itemize}
	\item Future Tasks (To be completed after October Beamtime)
	\begin{itemize}
		\item Quantify how errors in mounting affect results (August 2010)
	\end{itemize}
\end{itemize}
\subsubsection*{Overview}

The samples have already been measured using MicroCT at the Insitute for Biomechanics, and are thus already dissected, cleaned, labeled, and waiting in ethanol. This means the preparation step only involves picking the samples up from the IfB and mounting them to the Robot Sample Mounts.

Since a goniometer will be installed at the beam, the mounting need not be precise. The bones will be mounted to the metal platforms using melted wax. They will then be aligned to be mostly vertical by eye and then loaded onto the tray.

\subsection{Operation During Shifts : Automation}
\subsubsection*{Timeline}
\begin{itemize}
 	\item Completed Tasks
	\begin{itemize}
		\item Robot Sample Exchange 
		\item Parameter Entry and Scan Control
		\item Database entry creation and parameter storage
	\end{itemize}
	\item Current Tasks (Expected to be operational by October Beamtime)
	\begin{itemize}
		\item Sample Alignment Algorithms
		\item Center Detection
	\end{itemize}
	\item Future Tasks (To be completed after October Beamtime)
	\begin{itemize}
		\item Quality Assesment (March 2010)
		\item Reconstruction Initiation (January 2010)
		\item Data Transfer (December 2010)
	\end{itemize}
\end{itemize}
\subsubsection*{Overview}
As part of the collaboration with the Institute for Biomechanics at the ETH Zurich, I am a member of the TOMCAT Tomography group. Providing assistance in running extended beamtimes (over 1 day) to allow reasonable working/sleeping hours minimizing the liklihood of error due to operator exhaustion. Though we expect to have most of the process automated in time for the first beamtime, several steps such as starting reconstructions, verifying scan quality, and changing trays still require some degree of user-intervention


For the experiment we will be utilizing the unique high-throughput end-station at TOMCAT. The sample exchange aspect of the station has been successfully tested on several experiments with very promising results [8]. The first 200 samples will serve as a batch to test the entire automated measurement process as a whole and to optimize the automatic reconstruction and quantification of such a large sample set.
Specifically, we wish to assess the murine intracortical porosity through the canal network and the osteocyte lacunar system, both factors in overall bone quality, which are strongly linked to genetic background. We will assess the bone at 56\% of the entire femur length (consistent with previous studies) at a pixel size of 740 nm at the anterior, posterior, lateral, and medial sites. The choice of sites allows for a measure of compensation for the anisotropy of bone structure. The tomographic scans will be conducted at 17.5 keV with a 100 ms exposure time for 2000 projections twice binned. The results will then be quantitatively analyzed using the same negative imaging technique and morphometric analysis developed in the preliminary study [1]. Principally we will be measuring previously established parameters such as volume density, number density, and mean volume of a single canal branch or osteocyte lacuna in both the canal network and the osteocyte lacunar system.



