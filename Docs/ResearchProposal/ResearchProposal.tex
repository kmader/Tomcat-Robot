\documentclass[a4paper,10pt]{article}
\usepackage{amsmath, amssymb, graphics}
\usepackage{color,cite,graphicx}

\newcommand{\mathsym}[1]{{}}
\newcommand{\unicode}{{}}



%opening
\title{PhD Research Proposal : High Throughput Ultrastructural Phenotype Study of Mouse Femurs}
\author{Kevin Mader}

\begin{document}

\maketitle

\begin{abstract}
The aim of the project is to asses the ultrastructure of bone on a genome-wide scale on the femur bones from 2000 mice. Specifically the vascular network and the osteocyte lacunar system within trabecular and cortical bone will be quantitatively assessed using proven imaging strategies. This experiment will be the assessing the first 200 samples of this group.
\end{abstract}
\tableofcontents
\newpage


\section{Project Overview}
\subsection{Summary}
The goal of the project is to assess the ultrastructure of the femur bones from a set of 2000 mice, carefully bred and raised to provide a genome-wide sampling with minimal variation in environmental factors. Thus the variations in observations will come from genetic differences. Specifically the vascular network and the osteocyte lacunar system within trabecular and cortical bone will be quantitatively assessed using proven imaging strategies. The aim of this particular experiment is to image the first 200 of these bone samples. This experiment will also serve as the first large scale study enabled by the new high-throughput tomograhy system developed at the Tomcat beamline which allows samples to be automatically loaded, aligned, and unloaded without user intervention in quantities of 60 at a time. Significantly reducing the potential for user error, and eliminating wasted time between experiments due to opening, re-mounting, re-aligning, and security checks. This study is in reality the extension of a previously done study at the Institute for Biomechanics looking at the same bone samples and attributing genetic influence at a much lower resolution using a MicroCT scanner with 50x larger pixel size. So while the specific parameters being investigated are different, the methods for genetic analysis will be identical. Additionally the previous study will allow for the parameters in our study to be validated where overlap exists enabling a very easy check on whether the samples were measured, and analyzed correctly. 
\subsection{Biological Background}
Osteoporosis leads to higher risk of bone fracture through decreased bone mass and architectural changes leading to decreased bone quality [3]. While bone mineral density (BMD) is the most important factor in assessing fracture risk, utilizing morphological parameters of cortical bone can significantly increase the predictive power [4,5]. Specifically with cortical bone, BMD has been shown to be a weak factor in predicting mechanical properties, while microarchitectural parameters and porosity measurements are much better indicators for strength [6,7]. The most problematic of the fractures incurred through bone weakness is the proximal femur fracture. Thus the femur bone from mice will be analyzed.
A previous study has shown a strong connection between genetic background and morphology of bone samples, but has only done so on the scale of a few animals [1]. This study, however, is insufficient to determine the links between genetic background and bone ultrastructure. A genome-wide set of inbred mice has been created at The Jackson Laboratory (Bay Harbor, Maine, USA). The mice were inbred for many generations to ensure the genetic purity of each sample. Furthermore the mice were raised in an identical environment to minimize the effect environmental factors. The femurs have been extracted from these mice and reside at the Institute for Biomechanics at the ETH.  
We believe this first of its kind study will lead to revolutions in the understanding of genetic regulation of bone structure and bring more complete information to the studies of mechanism in genetic disorders and diseases in bone. Furthermore the information will provide a quantitative backing to re-examine previously purely epidemiological studies revealing new possibilities for treatment and prevention. Against osteoporosis we expect to lay the foundation for identifying new targets for treatment, and particularly at risk populations and provide the tools to much more accurately estimate the efficacy of treatments based on genetic background.


\begin{itemize}
 	\item Canal Interest
	\begin{itemize}
		\item Establish baseline, homeostatic parameters for canals and canal networks
		\item Correlate the variation in these parameters to macroscopic observed phenomenon (bone strength, bone density, …)
	\end{itemize}
	\item Lacunae Interest
	\begin{itemize}
		\item Establish baseline, homeostatic parameters for lacunae
		\item Correlate the variation in these parameters to macroscopic observed phenomenon (bone strength, bone density, …)
	\end{itemize}
	\item Outlook
	\begin{itemize}
		\item Some myleoma cells embed themselves in bone-matrix and become immortal and untreatable through techniques, further understanding of the behavior of osteocytes and how they interact with the matrix could be useful in developing new treatments to differentiate between these cells and carcinogenic cells.
	\end{itemize}
\end{itemize}








\newpage
\section{Pre-measuement Tasks}

\subsection{Robot for Automatic Sample Exchange}
\subsubsection*{Timeline}
\begin{itemize}
 	\item Completed Tasks
	\begin{itemize}
		\item Robot Control Software
		\item Epics Integration
		\item Operation with 20 Samples
	\end{itemize}
	\item Current Tasks (Expected to be operational by October Beamtime)
	\begin{itemize}
		\item Robot Expansion to 60 Samples
	\end{itemize}
	\item Future Tasks
	\begin{itemize}
		\item Beamline Integration with ESCAPE System to simplify future changes, and new components (January 2010)
		\item Quantification of Robot and Stage Accuracy (March 2010)
	\end{itemize}
\end{itemize}
\subsubsection*{Overview}
The first task completed as part of this research project is the implementation of the Automatic Sample Exchange Robot at the TOMCAT beamline. The ability to conduct large-scale studies such as this requires an automated system, not only to minimize the human-time required, but also to significantly accelerate the process as the safety procedures  involved with human operation such as opening and closing the door, conducting security searches, and so forth are only required to exchange the tray every 60 samples rather than for every single sample

\subsubsection*{Epics Overview}
EPICS (Experimental Physics Industrial Control System : http://www.aps.anl.gov/epics/) is the standard for beamline control at the Swiss Light Source. All instrumentation control, automation, and user interaction is conducted through it. The basic role of Epics is to allow control of any instrument from any computer. An Epics driver is written for each instrument which provides access to the instrument through a series of public channels. The machine connected to the device is refered to as an IOC. With most instrumentation this is a machine physically wired to the piece of instrumentation. When physical wires are not involved the machine, and thus the machine could be running anywhere the machine is refered to as a Soft-IOC. The Soft-IOC's are not computers themselves but are rather run under virtual machines where many are run from a single computer. These channels can transmit and receive information from the device. They are also capable of performing computation, and several other more complicated tasks. The GUI applications send and receive information from the channels which in turn communicate with the actual instruments. 

The Sample Exchange (Robot) is a Staubli 4-Axis RX60 Robot system. The robot serves to move samples from a tray to the measurement stage on the beamline. The robot is integrated into the beamline environment using VAL3 code communicating through a network interface to an Epics Soft-IOC (controller computer). The Soft-IOC then serves to provide control of the robot through channels (global variables broadcast over the network). The. The first step was to migrate the robot control software into the EPICS control framework utilized at the SLS for interfacing external instruments with beamline components. This migration not only served to make the robot behavior more predictable and tolerant of computer crashes but also allows other services to interact with the robot in the same manner as they would with a stage or shutter. This was done by writing an interface using the robot's VAL3 programming language to accept connections from other devices on the network. Then a driver was written to connect to the robot and translate commands to accessible channels in EPICS. The next step was to develop a coherent interface for selecting, naming, and positioning samples. The software was developed in Python and scales to whatever number of samples are needed. The software serves as a front-end to everything on the beamline in order to protect against potential collisions. The tool was designed in a modular way to be easily adaptable to updates and changes in the beamline setup. Additionally, a scripting language was developed to allow complex management of many samples. Specifically this language will prove useful for time progression experiments 
Although no results on the genetic project have yet been obtained, the robot system was tested on cement, bone, and brains samples. With few exceptions the tests went very well and proved that the robot will be a viable source for accurately changing samples over the course of 8 hour or longer experiments.
\begin{center}
% use packages: array
\begin{tabular}{|l|l|l|}
Performance Comparison & By Hand & With Robot \\
Sample Loading Time & 5 min & 0.5 sec \\
Sample Alignment Time & 3 minutes & 3 minutes\\
Intervention-free Operation Time & 30 minutes/1 scan & 12 hours
\end{tabular}
\end{center}

\subsection{Goniometer for Sample Orientation Flexibility}
\subsubsection*{Timeline}
\begin{itemize}
 	\item Completed Tasks
	\item Current Tasks (Expected to be operational by October Beamtime)
	\begin{itemize}
		\item Installation of Goniometer
		\item Calibration of Goniometer
		\item Integration of Goniometer with Alignment Software
	\end{itemize}
	\item Future Tasks
	\begin{itemize}
		\item Beamline Integration with ESCAPE System to simplify future changes, and new components (January 2010)
	\end{itemize}
\end{itemize}
\subsubsection*{Overview}
In order for the measurements to be comparable the position and orientation of the sample must be consistent. This is conventionally done by manually mounting the sample using a microscope-based setup and spending several minutes ensuring the sample is straight in all directions. This appears to be sufficiently accurate, but it is quite time-consuming and is highly sensitive to human error. We have, therefore, devised a method to automatically orient the samples using a Goniometer. This will significantly reduce the preparation time and should lead to more consistent results than could be obtained by hand. The most time-consuming step of alignment

\subsection{Script for automatic sample centering, orienting, and ROI detection}
\subsubsection*{Timeline}
\begin{itemize}
 	\item Completed Tasks
	\begin{itemize}
		\item Thresholding
		\item Fitting Cylindrical Sample to find tilt and offset
	\end{itemize}
	\item Current Tasks (Expected to be operational by October Beamtime)
	\begin{itemize}
		\item Camera Control Software for Snapshots
		\item Finding Top and Bottom of Samples and Moving to 56/%
		\item Safe movement of stages to center sample
	\end{itemize}
	\item Future Tasks
	\begin{itemize}
		\item Increasing flexibility to allow for more arbitrary sample types (August 2010)
		\item Self-identification of missing or poorly mounted sample (March 2010)
		\item Quantification of Accuracy of Alignment Algorithm (March 2010)
	\end{itemize}
\end{itemize}
\subsubsection*{Overview}

Several scripts have already been written for automatic sample detection, center finding, and orientation. The basic premise for their operation is outlined below
\begin{itemize}
 	\item Detection 
	\begin{itemize}
		\item Materials Model - Projections are classified into these three categories
		\item Air, glue, mounting wax are significantly less dense than bone
		\item Metal is significantly more dense than bone
	\end{itemize}
	\item Centering
	\begin{itemize}
		\item To center the sample the Bone region is selected and a curve is fit through the center giving the center of mass of the bone in X as a function of Y
		\item The stage is then moved until the offset is nearly  zero at 0,90,180 degrees
	\end{itemize}
	\item Alignment
	\begin{itemize}
		\item To align the sample the same data is used only this time it is found
	\end{itemize}
\end{itemize}


\subsection{Organizing Sample Names and Genetic Background in Database}
\subsubsection*{Timeline}
\begin{itemize}
 	\item Completed Tasks
	\begin{itemize}
		\item Beamline Database Setup and Installation
		\item Integration with Robot and Seqeuncer Software
		\item Web-based interface for searching and browsing
		\item Interface from OpenVMS system
	\end{itemize}
	\item Current Tasks (Expected to be operational by October Beamtime)
	\begin{itemize}
		\item Getting Sample Names, Background, and Parameters from Original Study
	\end{itemize}
	\item Future Tasks
\end{itemize}
\subsubsection*{Overview}
The idea behind the database is to have one location where every piece of information about each sample can be found from alignment position and genetic background to quantitive parameters

\subsection{Applications for Beamtime}
\subsubsection*{Timeline}
\begin{itemize}
 	\item Completed Tasks
	\begin{itemize}
		\item Application for First Beamtime
	\end{itemize}
	\item Current Tasks (Expected to be operational by October Beamtime)
	\item Future Tasks (To be completed after October Beamtime)
	\begin{itemize}
		\item Submission of Report from First Beamtime (January 2010)
		\item Long-Term Beamline Proposal (March 2010)
	\end{itemize}
\end{itemize}
The first application for formal beamtime has been submitted and was granted. While much of the development work can be done during internal shifts, the project itself is too large to be done entirely within these shifts. Based on the current estimates of being able to measure a tray of 60 samples in 13.25 hours, and realistically being able to measure 4 trays during one beamtime (due to mounting, storage, and analysis constraints it does not make sense to attempt more) it will take around 9 sessions to be able to measure all of the samples.

The estimate for the beamtime required is based on each tomogram taking 15 minutes. We will then take 2 tomograms per sample. The loading and unloading time for each sample with the robot setup is 30 seconds. Finally to automatically align the sample and change objectives will be 2 minutes per sample. Thus we expect : Samples (200) * Time Per Scan ( Time per region (3.3 min)*4 + Changing Time (1 min)+Alignment (2 min) ) / Minutes Per Shift (60 * 8) = 6.75 shifts. We request an additional two shifts in the event of specific samples that require re-imaging or technical difficulties. In summary, we apply for 9 shifts.

For the rest of the experiment it will be nessecary to apply for a longer term beamline proposal (for the remaining 1800 samples); however, for this we must first submit the initial results in a report demonstrating that our experiment works and we are capable of continuing the study. In the event a longer term proposal is not accepted it is still possible to finish using a number of shorter beamtimes, but that is a less ideal scenerio.


\newpage
\section{Measurement}


\subsection{Sample Preparation}
\subsubsection*{Timeline}
\begin{itemize}
 	\item Completed Tasks
	\begin{itemize}
		\item Mount and measure test femur
	\end{itemize}
	\item Current Tasks (Expected to be operational by October Beamtime)
	\begin{itemize}
		\item Mount and measure the extreme cases to ensure measurement process works
		\item Determine maximum error in mounting where alignment algorithms and measurement still functions
		\item Glue and Mount 200 samples
	\end{itemize}
	\item Future Tasks (To be completed after October Beamtime)
	\begin{itemize}
		\item Quantify how errors in mounting affect results (August 2010)
	\end{itemize}
\end{itemize}
\subsubsection*{Overview}

The samples have already been measured using MicroCT at the Insitute for Biomechanics, and are thus already dissected, cleaned, labeled, and waiting in ethanol. This means the preparation step only involves picking the samples up from the IfB and mounting them to the Robot Sample Mounts.

Since a goniometer will be installed at the beam, the mounting need not be precise. The bones will be mounted to the metal platforms using melted wax. They will then be aligned to be mostly vertical by eye and then loaded onto the tray.

\subsection{Operation During Shifts : Automation}
\subsubsection*{Timeline}
\begin{itemize}
 	\item Completed Tasks
	\begin{itemize}
		\item Robot Sample Exchange 
		\item Parameter Entry and Scan Control
		\item Database entry creation and parameter storage
	\end{itemize}
	\item Current Tasks (Expected to be operational by October Beamtime)
	\begin{itemize}
		\item Sample Alignment Algorithms
		\item Center Detection
	\end{itemize}
	\item Future Tasks (To be completed after October Beamtime)
	\begin{itemize}
		\item Quality Assesment (March 2010)
		\item Reconstruction Initiation (January 2010)
		\item Data Transfer (December 2010)
	\end{itemize}
\end{itemize}
\subsubsection*{Overview}
As part of the collaboration with the Institute for Biomechanics at the ETH Zurich, I am a member of the TOMCAT Tomography group. Providing assistance in running extended beamtimes (over 1 day) to allow reasonable working/sleeping hours minimizing the liklihood of error due to operator exhaustion. Though we expect to have most of the process automated in time for the first beamtime, several steps such as starting reconstructions, verifying scan quality, and changing trays still require some degree of user-intervention


For the experiment we will be utilizing the unique high-throughput end-station at TOMCAT. The sample exchange aspect of the station has been successfully tested on several experiments with very promising results [8]. The first 200 samples will serve as a batch to test the entire automated measurement process as a whole and to optimize the automatic reconstruction and quantification of such a large sample set.
Specifically, we wish to assess the murine intracortical porosity through the canal network and the osteocyte lacunar system, both factors in overall bone quality, which are strongly linked to genetic background. We will assess the bone at 56\% of the entire femur length (consistent with previous studies) at a pixel size of 740 nm at the anterior, posterior, lateral, and medial sites. The choice of sites allows for a measure of compensation for the anisotropy of bone structure. The tomographic scans will be conducted at 17.5 keV with a 100 ms exposure time for 2000 projections twice binned. The results will then be quantitatively analyzed using the same negative imaging technique and morphometric analysis developed in the preliminary study [1]. Principally we will be measuring previously established parameters such as volume density, number density, and mean volume of a single canal branch or osteocyte lacuna in both the canal network and the osteocyte lacunar system.





\newpage
\section{Post-Measurement Analysis}
\subsection{Automated Center Finding and reconstruction}
\subsubsection*{Timeline}
\begin{itemize}
 	\item Completed Tasks
	\begin{itemize}
		\item Implement Algorithm to Find Center based on Entropy
		\item Test Algorithm on Bone Samples
	\end{itemize}
	\item Current Tasks (Expected to be operational by October Beamtime)
	\begin{itemize}
		\item Tune Algorithm Parameters for Femur Samples
	\end{itemize}
	\item Future Tasks (To be completed after October Beamtime)
	\begin{itemize}
		\item Automatic Quality Assesment (January 2010)
	\end{itemize}
\end{itemize}

Some of the code for automatic center determination has been created, but it is not currently in a state where it is able to be blindly run on data as several parameters require tweaking. The problem will be greatly simplified through the automatic alignment which will  ensure the center of rotation is very close to the center pixel and the sample is properly oriented.

\subsection{Correction for Local Tomography Artifacts}
\subsubsection*{Timeline}
\begin{itemize}
 	\item Completed Tasks
	\begin{itemize}
		\item Development of Algorithm
		\item Test on Simulated Datasets
		\item Test on Aluminum Cylinder Samples

	\end{itemize}
	\item Current Tasks (Expected to be operational by October Beamtime)
	\begin{itemize}
		\item Test on Bone Samples
		\item Test on Embedded Samples
		\item Compare results to other algorithms
		\item Publish Results
	\end{itemize}
	\item Future Tasks (To be completed after October Beamtime)
	\begin{itemize}
		\item Implementation of Reconstruction in Standard Beamline Pipeline (August 2010)
	\end{itemize}
\end{itemize}
For the samples we are imaging we will be doing local tomography. Meaning the samples will be larger than the field of view. One of the basic assumptions in Tomography reconstruction is that the entire object be within the field of view in all the projections taken. When the entire object is not visible, there is missing data. Without these data it is impossible to reconstruct perfectly. Additionally using standard algorithms, the region outside the field of view will create artifacts in the gray-values of the reconstruction. Giving the so-called ring or halo. When attempting to analyze the data, principally in the thresholding step, the artifact can cause regions that are not bone to appear as bone. As the primary interest in this project is the porosity this could have a significant effect on the end results causing less of the porosity (that closer to the edge of the field of view) to be visible. Particularly parameters such as $\dfrac{Porosity_{Volume}}{Total_{Volume}}$, and canal density could be greatly skewed. Since there are so many lacunae the results for them would, however, likely remain usable.
The basic idea for our reconstruction technique is to make strong assumption about what the object outside of the field of view looks like, and to then to tweak this assumption until it matches the observed results best. I have called the assumption the Plum-Pudding model as it brings to mind a very clear visual.

The plum-pudding model for local tomography implies that the object being measured can be approximated as a bath of relatively constant absoprtion material. Within this bath sit the plums, holes and bumps with higher absorption. This model is used because the primary assumption of this technique is that smooth radial gradients in intensity come entirely from data outside of the region of interest and are thus background. Sharp edges represent the actual signal, and boundaries between the bumps, holes, and bath.

Initial Draft of Local Tomography Paper in Appendix

\subsection{Loading Data to ETHZ IfB Cluster}
\subsubsection*{Timeline}
\begin{itemize}
 	\item Completed Tasks
	\begin{itemize}
		\item Test FTP pipeline between IfB and PSI
		\item Manually copy and register datasets
	\end{itemize}
	\item Current Tasks (Expected to be operational by October Beamtime)
	\begin{itemize}
		\item Setup scripts to copy all datasets to IfB as a background batch job
	\end{itemize}
	\item Future Tasks (To be completed after October Beamtime)
	\begin{itemize}
		\item Setup scripts to automatically copy and register samples in Database (March 2010)
	\end{itemize}
\end{itemize}

The procedure for transferring the data from PSI to the IfB cluster involves moving the data to the PSI FTP Archive Server using the existing scripts and then downloading and extracting this data using another script running on the cluster.
After the data is downloaded, the samples need to be registered into the ScanCo sample database and the project such that the analysis can be run in parallel with only a single command per batch and the results stored in the same location

\subsection{Mophological Analysis using XIPL Cluster}
\subsubsection*{Timeline}
\begin{itemize}
 	\item Completed Tasks
	\begin{itemize}
		\item Perform standard Canal metrics on Cortical Bone Samples
		\item Measure femur and create new scripts to analyze our samples
	\end{itemize}
	\item Current Tasks (Expected to be operational by October Beamtime)
	\begin{itemize}
		\item Revise scripts used in original study to work on data with orders of magnitude higher resolution and only small subregion of bone
		\item Develop metrics for Lacuna analysis
	\end{itemize}
	\item Future Tasks (To be completed after October Beamtime)
	\begin{itemize}
		\item Analyze lacuna on existing data
		\item Publish Lacuna Metric Paper
	\end{itemize}
\end{itemize}
\subsubsection*{Overview}
All of the parameters being computed are morphological, meaning they are related to the shape of the bone and the shape of the holes in the bone. The data measured will be a stack of grayscale images, this cannot be directly analyzed and must first be converted into a black and white image where black or 0 is empty space and white or 1 is bone. This is done over several steps, the first being a filtering. The filtering smooths out the noise and artifacts. After filtering an adaptive threshold is run. This automatically finds the threshold values for bone and air by searching for a minima in the intensity histogram. After thresholding several open operations are run to remove small objects
\subsubsection*{Bone}
Within the bone or solid areas, we will be looking at parameters such as thickness (the thickness of the bone) and total volume of bone. These parameters have already been measured in the previous study, we do not expect to gain much new information, but the results will be much more precise (due to our higher resolution), and will serve as a cross-check to ensure samples were not mislabeled, broken, contaminated, misaligned or some other disturbance.
\subsubsection*{Porosity}
In the cortical bone, the porosity, or empty space, will be assessed. Looking at two different features. The first is canal networks, and the second is lacunae.
The canal networks will be assessed using the same tools developed to analyze vessel networks in vascular systems. Specific parameters that will be quantified are thickness, connectivity, density, and length.
\subsubsection*{Classification}
The problem of classification is not trivial because while it is easy to identify by eye what is a canal, what is a lacuna, and what is an artifact, it is significantly trickier to do in an automated fashion. The easiest way to classify objects is to group them by volume. While single lacuna are around 400-900 $\mu m^3$ canals are closer to the order of 27000 $\mu m^3$

\begin{itemize}
	\item Lacuna
\begin{figure}[htp]
 \centering
\begin{center}
 \includegraphics[width=0.9\textwidth]{Figures/TypicalLacuna.ps}
 % TypicalLacuna.ps: 156757072x156142432 pixel, 0dpi, infxinf cm, bb=
\end{center}
 \caption{Here is how a typical Lacuna looks. The lacuna volume is around (550 $\mu m^3$ and the dimensions are 15.4 x 46.2 x 127.4 $\mu$m}
 \label{fig:NormLacun}
\end{figure}	

\begin{figure}[htp]
 \centering
\begin{center}
 \includegraphics[width=0.9\textwidth]{Figures/JoinedLacunae.ps}
 % JoinedLacunae.ps: 1179666x1179666 pixel, 0dpi, infxinf cm, bb=
\end{center}

 \caption{It is often a problem that lacuna will touch at one point and thus register as one object. The object contains around 4800 voxels and the dimensions are 36.4 x 15.4 x 37.8 $\mu$m. This problem is however alleviated by performing an erosion before determining groups. However when they overlap more than this it is exceedingly difficult to seperate the spaces. }
 \label{fig:TwoLacun}
\end{figure}
	\begin{itemize}
		\item Anisotropy
The lacuna are generally non-isotropic, and it is known that they play a role in force sensing


		\item Orientation
Since the shape is not isotropic the cells have a primary axis, allowing for an orientation to be determined, from this information it will be possible to identify trends in the cell orientation such as sheets. The primary factor to be measured will be the coherence distance for orientation very similar to metrics used in optics for polarization. This distance will represent how well organized the cells are, a high number would mean there is a high degree of either planning or communication between these cells while a low number would suggest less of a trend. These factors could be important determinants of bone strength and potentially crucial factors in pathological bone as an early metric for identifying high-risk osteoporosis patients, cancer, and other similar diaseases.
	\end{itemize}
	\item Canal Network
\begin{figure}[htp]
 \centering
 \includegraphics[width=0.9\textwidth]{Figures/Canal.ps}
 % Canal.PS;1: 1179666x1179666 pixel, 0dpi, infxinf cm, bb=
 \caption{Here is how a typical canal looks. The canal contains around 10000 voxels, and the dimensions are 81.2 x 120.4 x 302.4 $\mu$m}
 \label{fig:NormCanal}
\end{figure}	
	\begin{itemize}
		\item Orientation
Since the shape is not isotropic the cells have a primary axis, allowing for an orientation to be determined, from this information it will be possible to identify trends in the cell orientation such as sheets. The primary factor to be measured will be the coherence distance for orientation very similar to metrics used in optics for polarization. This distance will represent how well organized the cells are, a high number would mean there is a high degree of either planning or communication between these cells while a low number would suggest less of a trend. These factors could be important determinants of bone strength and potentially crucial factors in pathological bone as an early metric for identifying high-risk osteoporosis patients, cancer, and other similar diaseases.
	\end{itemize}
\end{itemize}


\subsection{Automatically Register Results in Sample Database}
\subsubsection*{Timeline}
\begin{itemize}
 	\item Completed Tasks
	\begin{itemize}
		\item Interface between OpenVMS Cluster and PSI Sample Database
	\end{itemize}
	\item Current Tasks (Expected to be operational by October Beamtime)
	\begin{itemize}
		\item Integration of OpenVMS analysis to Sample Database
	\end{itemize}
	\item Future Tasks (To be completed after October Beamtime)
	\begin{itemize}
		\item Analyze lacuna on existing data
		\item Publish Lacuna Metric Paper
	\end{itemize}
\end{itemize}
\subsubsection*{Overview}
As stated earlier the database plays a critical role in keeping track of the progress of each sample (measured, reconstructed, analyzed), as well as having one place where all the relavant information for a sample can be found to make the analysis at the end significantly easier. It furthermore serves as the basis for future experiments of such scale in order to avoid a convoluted, disorganized system.
\subsection{Genetic Loci Analysis on Parameters}
\subsubsection*{Timeline}
\begin{itemize}
 	\item Completed Tasks
	\begin{itemize}
		\item Read previous study and examine analytical methods used
	\end{itemize}
	\item Current Tasks (Expected to be operational by October Beamtime)
	\item Future Tasks (To be completed after October Beamtime)
	\begin{itemize}
		\item Use model established in earlier study to analyze parameters obtained
	\end{itemize}
\end{itemize}
\subsubsection*{Overview}


\section{Conclusion}
As seen in the above report a majority of the development work for the project should be done on or around the completion of the first beamtime, with only minor additional tasks left to be performed over the following months. The extra-time will allow ample time to conduct all the measurements and remeasuring in the event of other problems. After all the samples are measured which would ideally be completed by the end of year 2, there will be plenty of time to analyze the results, publish and possibly begin investigatory studies into other aspects of bone structure for example looking into mineralization density and variations and the genetic influences there. The methods and tools developed to do this study will enable future studies of such scale to be done much quicker, and establish a standard pipeline where further genetic and other large sample number studies could be done in several beamtimes with minimal user intervention and without the need to develop new tools.
\end{document}
